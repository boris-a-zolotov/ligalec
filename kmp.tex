\documentclass[11pt,aspectratio=169,svgnames]{beamer}

\usefonttheme{professionalfonts}

\usepackage{amsmath,amssymb,amsthm,mathtools}
\usepackage{xcolor,graphicx,makecell}
\usepackage[russian]{babel}

\usepackage{tikz}
\usetikzlibrary{calc, arrows, arrows.meta}

\definecolor{dgray}{RGB}{15,15,15}
\definecolor{dplot}{HTML}{96e6ff}
\definecolor{ltr1}{HTML}{8ec1f5}
\definecolor{ltr2}{HTML}{fac293}
\definecolor{ltr3}{HTML}{ffc4ed}

\usebackgroundtemplate{%
	\includegraphics[width=\paperwidth,height=\paperheight]{img/lile-back-dgray}%
}

\setbeamersize{text margin left=12mm,text margin right=12mm}
\addtolength{\headsep}{0.55cm}
\setbeamertemplate{frametitle}[default][left,leftskip=0.85cm]
\setbeamertemplate{navigation symbols}{}
\setbeamertemplate{blocks}[rounded]
\setbeamertemplate{footline}{\vspace{1.4cm}}

\setbeamercolor{titlelike}{fg=white}
\setbeamercolor{normal text}{fg=white}
\setbeamercolor{block title}{bg=white!27!dgray,fg=white}
\setbeamercolor{block body}{bg=white!10!dgray}

\DeclarePairedDelimiter{\lr}{(}{)}
\DeclarePairedDelimiter{\atdeg}{[}{]}
\DeclarePairedDelimiter{\len}{|}{|}
\newcommand{\br}{\mathbb{R}}
\tikzset{>={Latex[width=1.2mm,length=2.8mm]}}

\usepackage{mathspec}

\setsansfont[
	Path = f/,
	Extension = .otf,
	BoldFont=fb,
	BoldItalicFont=fbi
		]{f}
		
\setmathfont(Digits)[Path = f/]{roboto.ttf}
\setmathfont(Latin)[Path = f/]{robotoi.ttf}
\setmathfont(Greek)[Path = f/, Uppercase]{roboto.ttf}
\setmathfont(Greek)[Path = f/, Lowercase]{robotoi.ttf}


\newenvironment{nblock}[1]{
	\begin{center} \begin{columns}[t] \begin{column}{110mm} \begin{block}{#1}
   }{
	\end{block} \end{column} \end{columns} \end{center}
}

\newcommand{\rthtd}[1]{
    node[right,text height = 3ex,text depth = 1ex]{#1}
}

\newcommand{\divsby}{
   \mathrel{\rlap{.}\rlap{\raisebox{0.55ex}{.}}\raisebox{1.1ex}{.}}}

\newcommand{\incg}[1]{\includegraphics[width=0.88\textwidth]{#1}}
\newcommand{\kmpstext}{Кнут — Моррис — Пратт; пример}

\newcommand{\proj}[3]{(0.97 * #1 cm + 0.72 * #2 cm,
                       #3 cm + 0.35 * #2 cm - 0.24254 * #1 cm)}

   \title{Задача поиска подстроки в строке}
   \date{\today}
   \author{Золотов Борис Алексеевич, аспирант МКН СПбГУ, \\ преподаватель ЛНМО}
   \institute{«Лига Лекторов», 3 сезон, полуфинал}

\begin{document} \maketitle


\section{substring-problem}

\begin{frame} \frametitle{Задача поиска подстроки в строке}
	Дан {\itshape\bfseries текст} \(T\) и {\itshape\bfseries образец} \(S\); проверить,
	содержится ли \(S\) в \(T\) —\\
	как «кад» содержится в «абракадабра». \bigskip\bigskip \pause

	Вы скажете: я просто нажимаю\ \ \begin{tikzpicture}%
	   [baseline={([yshift=-0.75ex]current bounding box.center)}]
	      \node[rectangle,draw=white,rounded corners=0.15cm,
	         text height=1.7ex,text depth=0.2ex,inner sep=0.2cm] at (0,0) {Ctrl};
	      \node[text height=1.7ex,text depth=0.2ex,inner sep=0.2cm] at (0.9,0) {\(+\)};
	      \node[rectangle,draw=white,rounded corners=0.15cm,
	         text height=1.7ex,text depth=0.2ex,inner sep=0.2cm] at (1.6,0) {F};
	\end{tikzpicture} , в чём проблема? \bigskip\bigskip \pause

	Вы-то понятно, но что в этот момент думает компьютер? \bigskip \pause

	Будем считать, что текст {\bfseries\itshape очень длинный} \\
	и, возможно, дописывается прямо сейчас.
\end{frame}


\section{naive-search}

\begin{frame} \frametitle{Наивный алгоритм}
Начиная с каждой позиции в \(T\), сравнивать символы в \(T\) и \(S\);\\
если удаётся дойти до конца \(S\) — мы нашли вхождение. \bigskip

\begin{center} \tikz[xscale=0.8,yscale=0.65]{
	    \node[ltr1, opacity=1] at (0.000000,0.400000) {1};
    \node[ltr2, opacity=1] at (0.800000,0.400000) {2};
    \node[ltr1, opacity=1] at (1.600000,0.400000) {1};
    \node[ltr2, opacity=1] at (2.400000,0.400000) {2};
    \node[ltr1, opacity=1] at (3.200000,0.400000) {1};
    \node[ltr2, opacity=1] at (4.000000,0.400000) {2};
    \node[ltr1, opacity=1] at (4.800000,0.400000) {1};
    \node[ltr2, opacity=1] at (5.600000,0.400000) {2};
    \node[ltr1, opacity=1] at (6.400000,0.400000) {1};
    \node[ltr2, opacity=1] at (7.200000,0.400000) {2};
    \node[ltr1, opacity=1] at (0.000000,-0.800000) {1};
    \node[ltr2, opacity=1] at (0.800000,-0.800000) {2};
    \node[ltr1, opacity=1] at (1.600000,-0.800000) {1};
    \node[ltr2, opacity=1] at (2.400000,-0.800000) {2};
    \node[ltr3, opacity=0.45] at (3.200000,-0.800000) {3};
    \node[ltr1, opacity=0.45] at (0.800000,-1.600000) {1};
    \node[ltr2, opacity=0.45] at (1.600000,-1.600000) {2};
    \node[ltr1, opacity=0.45] at (2.400000,-1.600000) {1};
    \node[ltr2, opacity=0.45] at (3.200000,-1.600000) {2};
    \node[ltr3, opacity=0.45] at (4.000000,-1.600000) {3};
    \node[ltr1, opacity=1] at (1.600000,-2.400000) {1};
    \node[ltr2, opacity=1] at (2.400000,-2.400000) {2};
    \node[ltr1, opacity=1] at (3.200000,-2.400000) {1};
    \node[ltr2, opacity=1] at (4.000000,-2.400000) {2};
    \node[ltr3, opacity=0.45] at (4.800000,-2.400000) {3};
    \node[ltr1, opacity=0.45] at (2.400000,-3.200000) {1};
    \node[ltr2, opacity=0.45] at (3.200000,-3.200000) {2};
    \node[ltr1, opacity=0.45] at (4.000000,-3.200000) {1};
    \node[ltr2, opacity=0.45] at (4.800000,-3.200000) {2};
    \node[ltr3, opacity=0.45] at (5.600000,-3.200000) {3};
    \node[ltr1, opacity=1] at (3.200000,-4.000000) {1};
    \node[ltr2, opacity=1] at (4.000000,-4.000000) {2};
    \node[ltr1, opacity=1] at (4.800000,-4.000000) {1};
    \node[ltr2, opacity=1] at (5.600000,-4.000000) {2};
    \node[ltr3, opacity=0.45] at (6.400000,-4.000000) {3};
    \node[ltr1, opacity=0.45] at (4.000000,-4.800000) {1};
    \node[ltr2, opacity=0.45] at (4.800000,-4.800000) {2};
    \node[ltr1, opacity=0.45] at (5.600000,-4.800000) {1};
    \node[ltr2, opacity=0.45] at (6.400000,-4.800000) {2};
    \node[ltr3, opacity=0.45] at (7.200000,-4.800000) {3};

	\node at (-0.8, 0.4) {\(T=\)};
	\node at (-0.8, -0.8) {\(S=\)};
} \end{center}
\end{frame}


\begin{frame} \frametitle{Проблема наивного алгоритма}
В худшем случае придётся сделать примерно \(\len*{S} \cdot \len*{T}\) \\
индивидуальных сравнений символов — представляете, \\
просмотреть длинный текст много раз? \bigskip \pause

В идеале бы добиться, чтобы количество операций \\
составляло примерно \(\len*{S} + \len*{T}\).
\end{frame}


\begin{frame} \frametitle{Наука об алгоритмах}
В этом заключается суть {\itshape\bfseries науки об алгоритмах —} \\
с помощью умных наблюдений и правильной \\
последовательности выполнения добиваться \\
{\itshape\bfseries оптимального} времени работы. \bigskip

Классические задачи: сортировка списка чисел, поиск\\
в упорядоченном наборе, поиск путей (в т. ч. на карте),\ldots
\end{frame}


\section{hash}

\begin{frame} \frametitle{Хэш}
Проблема: чтобы выяснить, что \(S\) не равно куску \(T\), \\
нужно много сравнений. \bigskip

Что, если сопоставить строке число, которое будет \\
характеризовать строку в целом и почти всегда \\
различаться для различных строк? \\
Такое число называется {\itshape\bfseries хэшом.}
\end{frame}


\begin{frame} \frametitle{Алгоритм Рабина — Карпа}
Значение хэша: многочлен, коэффициенты которого~—\\
символы строки; взять остаток от деления на простое число.\\
Пусть мы ищем подстроку \textcolor{ltr1}{1}\textcolor{ltr3}{3}\textcolor{ltr1}{1}:

  \textcolor{white}{\Large \[    
    \text{\textcolor{ltr1}{1}} \cdot 3^{2}\,+\,
    \text{\textcolor{ltr3}{3}} \cdot 3^{1}\,+\,
    \text{\textcolor{ltr1}{1}} \cdot 3^{0}\,=\,5 \pmod 7.
  \]} \bigskip

Последовательно считаем хэши кусков \(T\) нужной длины,\\
если хэш совпал с хэшом \(S\) — сравниваем посимвольно.
\end{frame}


\section{hash-example}

\begin{frame} \frametitle{Рабин — Карп; пересчёт хэша}
  \begin{center} \begin{tikzpicture}[xscale=0.71,yscale=0.87]
      \node[ltr3,right,text height=3ex,text depth=1ex] at (2 * 0, 1.5) {3};
  \node[ltr2,right,text height=3ex,text depth=1ex] at (2 * 1, 1.5) {2};
  \node[ltr2,right,text height=3ex,text depth=1ex] at (2 * 2, 1.5) {2};
  \node[ltr1,right,text height=3ex,text depth=1ex] at (2 * 3, 1.5) {1};
  \node[ltr3,right,text height=3ex,text depth=1ex] at (2 * 4, 1.5) {3};
  \node[ltr1,right,text height=3ex,text depth=1ex] at (2 * 5, 1.5) {1};
  \node[ltr1,right,text height=3ex,text depth=1ex] at (2 * 6, 1.5) {1};
  \draw (2 * 0, -0) \rthtd{\textcolor{white}{\( \text{\textcolor{ltr3}{3}} \cdot 3^{2} \)}};
  \draw (2 * 1, -0) \rthtd{\textcolor{white}{\( \text{\textcolor{ltr2}{2}} \cdot 3^{1} \)}};
  \draw (2 * 2, -0) \rthtd{\textcolor{white}{\( \text{\textcolor{ltr2}{2}} \cdot 3^{0} \)}};
  \draw (2 * 0 + 1.35, -0) \rthtd{\( + \)}
        (2 * 0 + 3.35, -0) \rthtd{\( + \)}
        (2 * 0 + 5.35, -0) \rthtd{\(= 0 \pmod 7 \)};
  \draw (2 * 1, -1) \rthtd{\textcolor{white}{\( \text{\textcolor{ltr2}{2}} \cdot 3^{2} \)}};
  \draw (2 * 2, -1) \rthtd{\textcolor{white}{\( \text{\textcolor{ltr2}{2}} \cdot 3^{1} \)}};
  \draw (2 * 3, -1) \rthtd{\textcolor{white}{\( \text{\textcolor{ltr1}{1}} \cdot 3^{0} \)}};
  \draw (2 * 1 + 1.35, -1) \rthtd{\( + \)}
        (2 * 1 + 3.35, -1) \rthtd{\( + \)}
        (2 * 1 + 5.35, -1) \rthtd{\(= 4 \pmod 7 \)};
  \draw (2 * 2, -2) \rthtd{\textcolor{white}{\( \text{\textcolor{ltr2}{2}} \cdot 3^{2} \)}};
  \draw (2 * 3, -2) \rthtd{\textcolor{white}{\( \text{\textcolor{ltr1}{1}} \cdot 3^{1} \)}};
  \draw (2 * 4, -2) \rthtd{\textcolor{white}{\( \text{\textcolor{ltr3}{3}} \cdot 3^{0} \)}};
  \draw (2 * 2 + 1.35, -2) \rthtd{\( + \)}
        (2 * 2 + 3.35, -2) \rthtd{\( + \)}
        (2 * 2 + 5.35, -2) \rthtd{\(= 3 \pmod 7 \)};
  \draw (2 * 3, -3) \rthtd{\textcolor{white}{\( \text{\textcolor{ltr1}{1}} \cdot 3^{2} \)}};
  \draw (2 * 4, -3) \rthtd{\textcolor{white}{\( \text{\textcolor{ltr3}{3}} \cdot 3^{1} \)}};
  \draw (2 * 5, -3) \rthtd{\textcolor{white}{\( \text{\textcolor{ltr1}{1}} \cdot 3^{0} \)}};
  \draw (2 * 3 + 1.35, -3) \rthtd{\( + \)}
        (2 * 3 + 3.35, -3) \rthtd{\( + \)}
        (2 * 3 + 5.35, -3) \rthtd{\(= 5 \pmod 7 \)};
  \draw (2 * 4, -4) \rthtd{\textcolor{white}{\( \text{\textcolor{ltr3}{3}} \cdot 3^{2} \)}};
  \draw (2 * 5, -4) \rthtd{\textcolor{white}{\( \text{\textcolor{ltr1}{1}} \cdot 3^{1} \)}};
  \draw (2 * 6, -4) \rthtd{\textcolor{white}{\( \text{\textcolor{ltr1}{1}} \cdot 3^{0} \)}};
  \draw (2 * 4 + 1.35, -4) \rthtd{\( + \)}
        (2 * 4 + 3.35, -4) \rthtd{\( + \)}
        (2 * 4 + 5.35, -4) \rthtd{\(= 3 \pmod 7 \)};

  \end{tikzpicture} \end{center}
\end{frame}


\begin{frame} \frametitle{Рабин — Карп; проблема совпадения хэшей}
  \begin{center} \begin{tikzpicture}[xscale=0.71,yscale=0.87]
    \input{python-kmp/rabin-karp-coinc}
  \end{tikzpicture} \end{center}
\end{frame}


\section{kmp}

\begin{frame} \frametitle{Алгоритм Кнута — Морриса — Пратта}
Попробуем снова посимвольно сравнивать \(S\) и \(T\),\\
но придумаем, как избегать ненужных сравнений,\\
если строка \(S\) правильно предобработана.\bigskip \pause

С каких позиций в \(T\) может начинаться \(S\)?
\end{frame}


\section{kmp-example}

\begin{frame} \frametitle{\kmpstext}
\begin{center} \begin{tikzpicture}[xscale=0.52]
    \draw[opacity=0] (2,0) -- (2,1.35);
    \node[ltr1] at (0.000000,0.000000) {1};
    \node[ltr2] at (1.000000,0.000000) {2};
    \node[ltr1] at (2.000000,0.000000) {1};
    \node[ltr3] at (3.000000,0.000000) {3};
    \node[ltr1] at (4.000000,0.000000) {1};
    \node[ltr3] at (5.000000,0.000000) {3};
    \node[ltr1] at (6.000000,0.000000) {1};
    \node[ltr2] at (7.000000,0.000000) {2};
    \node[ltr2] at (8.000000,0.000000) {2};
    \node[ltr1] at (9.000000,0.000000) {1};
    \node[ltr2] at (10.000000,0.000000) {2};
    \node[ltr1] at (11.000000,0.000000) {1};
    \node[ltr3] at (12.000000,0.000000) {3};
    \node[ltr1] at (13.000000,0.000000) {1};
    \node[ltr2] at (14.000000,0.000000) {2};
    \node[ltr1] at (15.000000,0.000000) {1};
    \node[ltr3] at (16.000000,0.000000) {3};
    \node[ltr1] at (17.000000,0.000000) {1};
    \node[ltr2] at (18.000000,0.000000) {2};
    \node[ltr1] at (19.000000,0.000000) {1};
    \node[ltr2] at (20.000000,0.000000) {2};
    \node[ltr1] at (21.000000,0.000000) {1};
    \node[ltr3] at (22.000000,0.000000) {3};
    \node[ltr1] at (0.000000,1.000000) {1};
    \node[ltr2] at (1.000000,1.000000) {2};
    \node[ltr1] at (2.000000,1.000000) {1};
    \node[ltr3] at (3.000000,1.000000) {3};
    \node[ltr1] at (4.000000,1.000000) {1};
    \node[ltr2] at (5.000000,1.000000) {2};
    \node[ltr1] at (6.000000,1.000000) {1};
    \node[ltr2] at (7.000000,1.000000) {2};
    \node[ltr1] at (8.000000,1.000000) {1};
    \node[ltr3] at (9.000000,1.000000) {3};
    \node[rotate=-90] at (0,0.5) {\(=\)};
    \node[rotate=-90] at (1,0.5) {\(=\)};
    \node[rotate=-90] at (2,0.5) {\(=\)};
    \node[rotate=-90] at (3,0.5) {\(=\)};
    \node[rotate=-90] at (4,0.5) {\(=\)};
    \node[rotate=-90] at (5,0.5) {\(\ne\)};
    \pause
    \draw[PaleGreen]
         (0,1.3) -- ++(-0.35,0) -- ++(0,-0.12) (0,1.3)
         -- ++(0,0) -- ++(0.35,0) -- ++(0,-0.12)
         (4,1.3) -- ++(0.35,0) -- ++(0,-0.12) (4,1.3)
         -- ++(-0,0) -- ++(-0.35,0) -- ++(0,-0.12);
\end{tikzpicture} \end{center} \end{frame}

\begin{frame} \frametitle{\kmpstext}
\begin{center} \begin{tikzpicture}[xscale=0.52]
    \draw[opacity=0] (2,0) -- (2,1.35);
    \node[ltr1] at (0.000000,0.000000) {1};
    \node[ltr2] at (1.000000,0.000000) {2};
    \node[ltr1] at (2.000000,0.000000) {1};
    \node[ltr3] at (3.000000,0.000000) {3};
    \node[ltr1] at (4.000000,0.000000) {1};
    \node[ltr3] at (5.000000,0.000000) {3};
    \node[ltr1] at (6.000000,0.000000) {1};
    \node[ltr2] at (7.000000,0.000000) {2};
    \node[ltr2] at (8.000000,0.000000) {2};
    \node[ltr1] at (9.000000,0.000000) {1};
    \node[ltr2] at (10.000000,0.000000) {2};
    \node[ltr1] at (11.000000,0.000000) {1};
    \node[ltr3] at (12.000000,0.000000) {3};
    \node[ltr1] at (13.000000,0.000000) {1};
    \node[ltr2] at (14.000000,0.000000) {2};
    \node[ltr1] at (15.000000,0.000000) {1};
    \node[ltr3] at (16.000000,0.000000) {3};
    \node[ltr1] at (17.000000,0.000000) {1};
    \node[ltr2] at (18.000000,0.000000) {2};
    \node[ltr1] at (19.000000,0.000000) {1};
    \node[ltr2] at (20.000000,0.000000) {2};
    \node[ltr1] at (21.000000,0.000000) {1};
    \node[ltr3] at (22.000000,0.000000) {3};
    \node[ltr1] at (4.000000,1.000000) {1};
    \node[ltr2] at (5.000000,1.000000) {2};
    \node[ltr1] at (6.000000,1.000000) {1};
    \node[ltr3] at (7.000000,1.000000) {3};
    \node[ltr1] at (8.000000,1.000000) {1};
    \node[ltr2] at (9.000000,1.000000) {2};
    \node[ltr1] at (10.000000,1.000000) {1};
    \node[ltr2] at (11.000000,1.000000) {2};
    \node[ltr1] at (12.000000,1.000000) {1};
    \node[ltr3] at (13.000000,1.000000) {3};
    \node[rotate=-90] at (5,0.5) {\(\ne\)};
\end{tikzpicture} \end{center} \end{frame}

\begin{frame} \frametitle{\kmpstext}
\begin{center} \begin{tikzpicture}[xscale=0.52]
    \draw[opacity=0] (2,0) -- (2,1.35);
    \node[ltr1] at (0.000000,0.000000) {1};
    \node[ltr2] at (1.000000,0.000000) {2};
    \node[ltr1] at (2.000000,0.000000) {1};
    \node[ltr3] at (3.000000,0.000000) {3};
    \node[ltr1] at (4.000000,0.000000) {1};
    \node[ltr3] at (5.000000,0.000000) {3};
    \node[ltr1] at (6.000000,0.000000) {1};
    \node[ltr2] at (7.000000,0.000000) {2};
    \node[ltr2] at (8.000000,0.000000) {2};
    \node[ltr1] at (9.000000,0.000000) {1};
    \node[ltr2] at (10.000000,0.000000) {2};
    \node[ltr1] at (11.000000,0.000000) {1};
    \node[ltr3] at (12.000000,0.000000) {3};
    \node[ltr1] at (13.000000,0.000000) {1};
    \node[ltr2] at (14.000000,0.000000) {2};
    \node[ltr1] at (15.000000,0.000000) {1};
    \node[ltr3] at (16.000000,0.000000) {3};
    \node[ltr1] at (17.000000,0.000000) {1};
    \node[ltr2] at (18.000000,0.000000) {2};
    \node[ltr1] at (19.000000,0.000000) {1};
    \node[ltr2] at (20.000000,0.000000) {2};
    \node[ltr1] at (21.000000,0.000000) {1};
    \node[ltr3] at (22.000000,0.000000) {3};
    \node[ltr1] at (9.000000,1.000000) {1};
    \node[ltr2] at (10.000000,1.000000) {2};
    \node[ltr1] at (11.000000,1.000000) {1};
    \node[ltr3] at (12.000000,1.000000) {3};
    \node[ltr1] at (13.000000,1.000000) {1};
    \node[ltr2] at (14.000000,1.000000) {2};
    \node[ltr1] at (15.000000,1.000000) {1};
    \node[ltr2] at (16.000000,1.000000) {2};
    \node[ltr1] at (17.000000,1.000000) {1};
    \node[ltr3] at (18.000000,1.000000) {3};
    \node[rotate=-90] at (9,0.5) {\(=\)};
    \node[rotate=-90] at (10,0.5) {\(=\)};
    \node[rotate=-90] at (11,0.5) {\(=\)};
    \node[rotate=-90] at (12,0.5) {\(=\)};
    \node[rotate=-90] at (13,0.5) {\(=\)};
    \node[rotate=-90] at (14,0.5) {\(=\)};
    \node[rotate=-90] at (15,0.5) {\(=\)};
    \node[rotate=-90] at (16,0.5) {\(\ne\)};
    \pause
    \draw[PaleGreen]
         (9,1.3) -- ++(-0.35,0) -- ++(0,-0.12) (9,1.3)
         -- ++(2,0) -- ++(0.35,0) -- ++(0,-0.12)
         (15,1.3) -- ++(0.35,0) -- ++(0,-0.12) (15,1.3)
         -- ++(-2,0) -- ++(-0.35,0) -- ++(0,-0.12);
\end{tikzpicture} \end{center} \end{frame}

\begin{frame} \frametitle{\kmpstext}
\begin{center} \begin{tikzpicture}[xscale=0.52]
    \draw[opacity=0] (2,0) -- (2,1.35);
    \node[ltr1] at (0.000000,0.000000) {1};
    \node[ltr2] at (1.000000,0.000000) {2};
    \node[ltr1] at (2.000000,0.000000) {1};
    \node[ltr3] at (3.000000,0.000000) {3};
    \node[ltr1] at (4.000000,0.000000) {1};
    \node[ltr3] at (5.000000,0.000000) {3};
    \node[ltr1] at (6.000000,0.000000) {1};
    \node[ltr2] at (7.000000,0.000000) {2};
    \node[ltr2] at (8.000000,0.000000) {2};
    \node[ltr1] at (9.000000,0.000000) {1};
    \node[ltr2] at (10.000000,0.000000) {2};
    \node[ltr1] at (11.000000,0.000000) {1};
    \node[ltr3] at (12.000000,0.000000) {3};
    \node[ltr1] at (13.000000,0.000000) {1};
    \node[ltr2] at (14.000000,0.000000) {2};
    \node[ltr1] at (15.000000,0.000000) {1};
    \node[ltr3] at (16.000000,0.000000) {3};
    \node[ltr1] at (17.000000,0.000000) {1};
    \node[ltr2] at (18.000000,0.000000) {2};
    \node[ltr1] at (19.000000,0.000000) {1};
    \node[ltr2] at (20.000000,0.000000) {2};
    \node[ltr1] at (21.000000,0.000000) {1};
    \node[ltr3] at (22.000000,0.000000) {3};
    \node[ltr1] at (13.000000,1.000000) {1};
    \node[ltr2] at (14.000000,1.000000) {2};
    \node[ltr1] at (15.000000,1.000000) {1};
    \node[ltr3] at (16.000000,1.000000) {3};
    \node[ltr1] at (17.000000,1.000000) {1};
    \node[ltr2] at (18.000000,1.000000) {2};
    \node[ltr1] at (19.000000,1.000000) {1};
    \node[ltr2] at (20.000000,1.000000) {2};
    \node[ltr1] at (21.000000,1.000000) {1};
    \node[ltr3] at (22.000000,1.000000) {3};
    \node[rotate=-90] at (16,0.5) {\(=\)};
    \node[rotate=-90] at (17,0.5) {\(=\)};
    \node[rotate=-90] at (18,0.5) {\(=\)};
    \node[rotate=-90] at (19,0.5) {\(=\)};
    \node[rotate=-90] at (20,0.5) {\(=\)};
    \node[rotate=-90] at (21,0.5) {\(=\)};
    \node[rotate=-90] at (22,0.5) {\(=\)};
\end{tikzpicture} \end{center} \end{frame}



\section{prefix-function}

\begin{frame} \frametitle{Префикс-функция}
Какая информация нам была нужна? Про каждую\\
позицию \(i\) строки \(S\) — наибольшая длина \(\ell< i\)\\
начального куска \(S\), который совпадает\\
с \(\ell\) символами перед позицией \(i\). \bigskip

{\itshape\bfseries Это} называется префикс-функцией; \(π(i)\). \bigskip

\begin{center} \begin{tikzpicture}[xscale=0.7]
	\node[ltr1] at (0,0) {\Large 1}; \node[ltr2] at (1,0) {\Large 2}; \node[ltr1] at (2,0) {\Large 1};
	\node[ltr3] at (3,0) {\Large 3}; \node[ltr1] at (4,0) {\Large 1}; \node[ltr2] at (5,0) {\Large 2};
	\node[ltr1] at (6,0) {\Large 1}; \node[ltr2] at (7,0) {\Large 2}; \node[ltr1] at (8,0) {\Large 1};
	\node[ltr3] at (9,0) {\Large 3}; \pause

	\node at (0.35,-0.35) {\scriptsize \color{Pink} 0};
	\node at (1.35,-0.35) {\scriptsize \color{Pink} 0};
	\node at (2.35,-0.35) {\scriptsize \color{Pink} 1};
	\node at (3.35,-0.35) {\scriptsize \color{Pink} 0};
	\node at (4.35,-0.35) {\scriptsize \color{Pink} 1};
	\node at (5.35,-0.35) {\scriptsize \color{Pink} 2};
	\node at (6.35,-0.35) {\scriptsize \color{Pink} 3}; \pause

	\draw[Crimson] (-0.35,0.25) -- (-0.35,0.4) -- (2.35,0.4) -- (2.35,0.25)
	   (3.65,0.25) -- (3.65,0.4) -- (6.35,0.4) -- (6.35,0.25); \pause

	\node at (7.35,-0.35) {\scriptsize \color{Pink} 2};
	\node at (8.35,-0.35) {\scriptsize \color{Pink} 3};
	\node at (9.35,-0.35) {\scriptsize \color{Pink} 4};
\end{tikzpicture} \end{center}

\end{frame}


\section{prefix-function-calculation}

\begin{frame} \frametitle{Вычисление префикс-функции}
\[ k = π(i);\qquad \begin{cases}
  π(i+1) = k+1, & s(i+1) = s(k+1), \\
  π(i+1) = 0, & k=0 \text{\ \ и\ \ } s(i+1) \ne s(1), \\
  k \coloneqq π(k), & \text{иначе.}
\end{cases} \] \bigskip

\begin{center} \begin{tikzpicture}[xscale=0.7]
	\node at (0,0) {\Large 1}; \node at (1,0) {\Large 2}; \node at (2,0) {\Large 1};
	\node at (3,0) {\Large 3}; \node at (4,0) {\Large 1}; \node at (5,0) {\Large 2};
	\node at (6,0) {\Large 1}; \node at (7,0) {\Large 2}; \node at (8,0) {\Large 1};
	\node at (9,0) {\Large 3};
	
	\draw[Plum,->] (6.45,0.35) -- (7.45,0.35);
	\node at (6.35,-0.35) {\scriptsize \color{Pink} 3}; \pause
	\draw[Plum,->] (6.35,0.35) to[out=150,in=30] (2.35,0.35);

	
	\draw[Plum,->] (2.45,0.35) -- (3.45,0.35)
		node[pos = 0.35]{\scriptsize \color{Plum} \(\times\)};
	\node at (2.35,-0.35) {\scriptsize \color{Pink} 1}; \pause	
	\draw[Plum,->] (2.35,0.35) to[out=150,in=30] (0.35,0.35);

	\draw[Plum,->] (0.45,0.35) -- (1.45,0.35); \pause
	\node at (7.35,-0.35) {\footnotesize \color{Pink} 2};
\end{tikzpicture} \end{center}

\end{frame}


\section{finite-automata}

\begin{frame} \frametitle{Конечные автоматы}
  Мы привели алгоритм распознавания строк, содержащих \(S\):
	\[*\,S\,*\] \bigskip

  Бывает полезно распознавать электронные адреса:
   \[
     [\text{нет @}]
      \text{\,@\,}
     [\text{нет точек, @}]\,.\,[\text{com}|\text{ru}|\ldots]
   \]
\end{frame}


\begin{frame} \frametitle{Конечные автоматы}
  Номера банковских карт: 12 цифр, но
   \[\sum \lr*{2 \cdot a_{2i}} \bmod 9 + a_{2i+1}\ \ \divsby\ \ 10.\] \bigskip

  Корректные даты, пароли, номера телефонов,\\
  пункты задач олимпиад…\\
  Вычислительная модель, эффективно распознающая\\
  такие строки~— {\itshape\bfseries конечный автомат.}
\end{frame}


\section{conclusion}

\begin{frame} \frametitle{\vspace*{-2.4cm}}
\begin{itemize}
	\item Для решения задачи поиска подстроки есть наивный алгоритм,\\
	 \(\len*{T} \cdot \len*{S}\) сравнений символов; \medskip
	\item Алгоритм Рабина — Карпа, \(\len*{T} + \len*{S}\) арифметических операций\\
	 и сравнений {\itshape\bfseries в среднем,}\\
	 но иногда хэши неравных подстрок совпадают; \medskip
	\item Алгоритм Кнута — Морриса — Пратта,\\
	 \(\len*{T} + \len*{S}\) сравнений символов.
\end{itemize} \bigskip

\begin{center}
	\Large\bf Спасибо за внимание!
\end{center}
\end{frame}

\end{document}



\begin{nblock}{\vspace*{-3ex}}
	Sample text
\end{nblock}
