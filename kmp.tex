\documentclass[11pt,aspectratio=169,svgnames]{beamer}

\usefonttheme{professionalfonts}

\usepackage{amsmath,amssymb,amsthm,mathtools}
\usepackage{xcolor,graphicx,makecell}
\usepackage[russian]{babel}

\usepackage{tikz}
\usetikzlibrary{calc, arrows, arrows.meta}

\definecolor{dgray}{RGB}{15,15,15}
\definecolor{dplot}{HTML}{96e6ff}

\usebackgroundtemplate{%
	\includegraphics[width=\paperwidth,height=\paperheight]{img/lile-back-dgray}%
}

\setbeamersize{text margin left=12mm,text margin right=12mm}
\addtolength{\headsep}{0.55cm}
\setbeamertemplate{frametitle}[default][left,leftskip=0.85cm]
\setbeamertemplate{navigation symbols}{}
\setbeamertemplate{blocks}[rounded]
\setbeamertemplate{footline}{\vspace{1.4cm}}

\setbeamercolor{titlelike}{fg=white}
\setbeamercolor{normal text}{fg=white}
\setbeamercolor{block title}{bg=white!27!dgray,fg=white}
\setbeamercolor{block body}{bg=white!10!dgray}

\DeclarePairedDelimiter{\lr}{(}{)}
\DeclarePairedDelimiter{\atdeg}{[}{]}
\DeclarePairedDelimiter{\len}{|}{|}
\newcommand{\br}{\mathbb{R}}
\tikzset{>={Latex[width=1.2mm,length=2.8mm]}}

\usepackage{mathspec}

\setsansfont[
	Path = f/,
	Extension = .otf,
	BoldFont=fb,
	BoldItalicFont=fbi
		]{f}
		
\setmathfont(Digits)[Path = f/]{roboto.ttf}
\setmathfont(Latin)[Path = f/]{robotoi.ttf}
\setmathfont(Greek)[Path = f/, Uppercase]{roboto.ttf}
\setmathfont(Greek)[Path = f/, Lowercase]{robotoi.ttf}


\newenvironment{nblock}[1]{
	\begin{center} \begin{columns}[t] \begin{column}{110mm} \begin{block}{#1}
   }{
	\end{block} \end{column} \end{columns} \end{center}
}

\newcommand{\incg}[1]{\includegraphics[width=0.88\textwidth]{#1}}

\newcommand{\proj}[3]{(0.97 * #1 cm + 0.72 * #2 cm,
                       #3 cm + 0.35 * #2 cm - 0.24254 * #1 cm)}

   \title{Задача поиска подстроки в строке}
   \date{\today}
   \author{Золотов Борис Алексеевич, аспирант МКН СПбГУ, \\ преподаватель ЛНМО}
   \institute{«Лига Лекторов», 3 сезон, полуфинал}

\begin{document} \maketitle

\begin{frame} \frametitle{Наивный алгоритм}
Начиная с каждой позиции в \(T\), сравнивать символы в \(T\) и \(S\);\\
если удаётся дойти до конца \(S\) — мы нашли вхождение. \bigskip

\begin{center} \tikz[xscale=0.8,yscale=0.65]{
	    \node[white] at (0.000000,0.400000) {1};
    \node[white] at (0.800000,0.400000) {2};
    \node[white] at (1.600000,0.400000) {1};
    \node[white] at (2.400000,0.400000) {2};
    \node[white] at (3.200000,0.400000) {1};
    \node[white] at (4.000000,0.400000) {2};
    \node[white] at (4.800000,0.400000) {1};
    \node[white] at (5.600000,0.400000) {2};
    \node[white] at (6.400000,0.400000) {1};
    \node[white] at (7.200000,0.400000) {2};
    \node[white] at (0.000000,-0.800000) {1};
    \node[white] at (0.800000,-0.800000) {2};
    \node[white] at (1.600000,-0.800000) {1};
    \node[white] at (2.400000,-0.800000) {2};
    \node[white,opacity=0.45] at (3.200000,-0.800000) {3};
    \node[white,opacity=0.45] at (0.800000,-1.600000) {1};
    \node[white,opacity=0.45] at (1.600000,-1.600000) {2};
    \node[white,opacity=0.45] at (2.400000,-1.600000) {1};
    \node[white,opacity=0.45] at (3.200000,-1.600000) {2};
    \node[white,opacity=0.45] at (4.000000,-1.600000) {3};
    \node[white] at (1.600000,-2.400000) {1};
    \node[white] at (2.400000,-2.400000) {2};
    \node[white] at (3.200000,-2.400000) {1};
    \node[white] at (4.000000,-2.400000) {2};
    \node[white,opacity=0.45] at (4.800000,-2.400000) {3};
    \node[white,opacity=0.45] at (2.400000,-3.200000) {1};
    \node[white,opacity=0.45] at (3.200000,-3.200000) {2};
    \node[white,opacity=0.45] at (4.000000,-3.200000) {1};
    \node[white,opacity=0.45] at (4.800000,-3.200000) {2};
    \node[white,opacity=0.45] at (5.600000,-3.200000) {3};
    \node[white] at (3.200000,-4.000000) {1};
    \node[white] at (4.000000,-4.000000) {2};
    \node[white] at (4.800000,-4.000000) {1};
    \node[white] at (5.600000,-4.000000) {2};
    \node[white,opacity=0.45] at (6.400000,-4.000000) {3};
    \node[white,opacity=0.45] at (4.000000,-4.800000) {1};
    \node[white,opacity=0.45] at (4.800000,-4.800000) {2};
    \node[white,opacity=0.45] at (5.600000,-4.800000) {1};
    \node[white,opacity=0.45] at (6.400000,-4.800000) {2};
    \node[white,opacity=0.45] at (7.200000,-4.800000) {3};

	\node at (-0.8, 0.4) {\(T=\)};
	\node at (-0.8, -0.8) {\(S=\)};
} \end{center}
\end{frame}


\begin{frame} \frametitle{Проблема наивного алгоритма}
В худшем случае придётся сделать примерно \(\len*{S} \cdot \len*{T}\) индивидуальных сравнений символов — представляете, просмотреть длинный текст несколько раз?
\end{frame}


\begin{frame} \frametitle{Префикс-функция}
Представим, что про каждую позицию \(i\) строки \(S\) мы знаем\\
наибольшую длину \(\ell< i\) начального куска \(S\), который\\
совпадает с \(\ell\) символами перед позицией \(i\). \bigskip

\begin{center} \begin{tikzpicture}[xscale=0.7]
	\node[ltr1] at (0,0) {\Large 1}; \node[ltr2] at (1,0) {\Large 2}; \node[ltr1] at (2,0) {\Large 1};
	\node[ltr3] at (3,0) {\Large 3}; \node[ltr1] at (4,0) {\Large 1}; \node[ltr2] at (5,0) {\Large 2};
	\node[ltr1] at (6,0) {\Large 1}; \node[ltr2] at (7,0) {\Large 2}; \node[ltr1] at (8,0) {\Large 1};
	\node[ltr3] at (9,0) {\Large 3}; \pause

	\node at (0.35,-0.35) {\scriptsize \color{Pink} 0};
	\node at (1.35,-0.35) {\scriptsize \color{Pink} 0};
	\node at (2.35,-0.35) {\scriptsize \color{Pink} 1};
	\node at (3.35,-0.35) {\scriptsize \color{Pink} 0};
	\node at (4.35,-0.35) {\scriptsize \color{Pink} 1};
	\node at (5.35,-0.35) {\scriptsize \color{Pink} 2};
	\node at (6.35,-0.35) {\scriptsize \color{Pink} 3}; \pause

	\draw[Crimson] (-0.35,0.25) -- (-0.35,0.4) -- (2.35,0.4) -- (2.35,0.25)
	   (3.65,0.25) -- (3.65,0.4) -- (6.35,0.4) -- (6.35,0.25); \pause

	\node at (7.35,-0.35) {\scriptsize \color{Pink} 2};
	\node at (8.35,-0.35) {\scriptsize \color{Pink} 3};
	\node at (9.35,-0.35) {\scriptsize \color{Pink} 4};
\end{tikzpicture} \end{center}
 \bigskip

Мы научимся находить эти числа, {\bfseries\itshape но потом.}
\end{frame}

\end{document}


\begin{frame} \frametitle{}
\end{frame}

\begin{nblock}{\vspace*{-3ex}}
	Sample text
\end{nblock}
