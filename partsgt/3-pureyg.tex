\section{Уступить или проехать (Ястребы и голуби)}

\newcommand{\yieldPO}{\gamePayoffs{-1}{-1}{1}{2}{2}{1}{-11}{-11}}


\begin{frame} \ft{Уступить или проехать: выигрыши}
	Оба уступают~— заминка на одном месте\\
	Один уступает~— оба счастливы\\
	Оба едут~— попадают в ДТП
\begin{center} \begin{tikzpicture}[scale=0.44]
	\gameLines
	\gameNames{Игрок 1}{Игрок 2}
	\gameActions{Уступить}{Ехать}
	\yieldPO
\end{tikzpicture} \end{center}
\end{frame}


\begin{frame} \ft{Доминирующая стратегия}
	Ни у одного из игроков нет\\
	доминирующей стратегии. (Игра симметрична,\\
	поэтому покажем только для первого.)
\begin{center} \begin{tikzpicture}[scale=0.44]
	\gameLines
	\gameNames{Игрок 1}{Игрок 2}
	\gameActions{Уступить}{Ехать}
	\yieldPO
	\obvod{-1.5}{3}{3.5}{3}{fill1}{0.5}{0}{\leqslant}
	\obvod{-1.5}{-1}{3.5}{-1}{fill1}{0.5}{0}{\geqslant}
\end{tikzpicture} \end{center}
\end{frame}


\begin{frame} \ft{Два равновесия Нэша и светофор}
	Есть два симметричных равновесия Нэша,\\
	от которых игрокам невыгодно отступать,\\
	если им указать, в какой они играют.
\begin{center} \begin{tikzpicture}[scale=0.44]
	\fill[p9480u,opacity=0.3] (0,0) rectangle (5,4)
		(0,0) rectangle (-5,-4);
	\gameLines
	\gameNames{Игрок 1}{Игрок 2}
	\gameActions{Уступить}{Ехать}
	\yieldPO
	\obvod{-1.5}{3}{3.5}{3}{fill1}{0.5}{0}{\leqslant}
	\obvod{1.5}{-3}{1.5}{1}{fill3}{0.5}{90}{\leqslant}
\end{tikzpicture} \end{center}
\end{frame}


\begin{frame} \ft{Два равновесия Нэша и светофор}
	Прибор, который это указывает, \\
	называется {\it светофор.} Но предлагается\\
	поискать равновесие ещё кое-где.
\begin{center} \begin{tikzpicture}[scale=0.44]
	\fill[p9480u,opacity=0.3] (0,0) rectangle (5,4)
		(0,0) rectangle (-5,-4);
	\gameLines
	\gameNames{Игрок 1}{Игрок 2}
	\gameActions{Уступить}{Ехать}
	\yieldPO
	\obvod{-1.5}{-1}{3.5}{-1}{fill1}{0.5}{0}{\geqslant}
	\obvod{-3.5}{-3}{-3.5}{1}{fill3}{0.5}{-90}{\leqslant}
\end{tikzpicture} \end{center}
\end{frame}
