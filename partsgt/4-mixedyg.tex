\section{Смешанные стратегии}

\begin{frame} \ft{Суть смешанных стратегий}
	Пусть {\it абстрактный коллективный} первый игрок\\
	уступает с вероятностью \(p\), а второй~— с вероятностью \(q\).\medskip

	Равновесие Нэша~— позиция, когда действие каждого\\
	игрока~— {\it лучший ответ} на действие другого.\medskip

	Найдём, какая \(q\) будет лучшим ответом\\
	в зависимости от \(p\).
\end{frame}


\begin{frame} \ft{Ожидаемый выигрыш второго игрока}
   \vphantom{t}\centerline{
	\(-1\cdot pq +2\cdot p\lr*{1-q} +1\cdot \lr*{1-p}q -11\cdot \lr*{1-p}\lr*{1-q} =\)} \\
   \vphantom{t}\centerline{
	\(= \lr*{12 - 15 p} \cdot q + 13 p - 11\)}\\
	Лучшая \(q\)~— либо 0, либо 1, либо {\it все возможные.}
\begin{center} \begin{tikzpicture}[scale=0.44]
	\gameLines
	\gameNames{Игрок 1}{Игрок 2}
	\gameActions{Уступить}{Ехать}
	\yieldPO
\end{tikzpicture} \end{center}
\end{frame}


\begin{frame} \ft{{\it Три} равновесия Нэша}
	\( p < \frac45 \), тогда \(q=1\)\\
	\( p = \frac45 \), тогда \(q\) любое, выигрыш от него не зависит\\
	\( p < \frac45 \), тогда \(q=0\)
\begin{center} \begin{tikzpicture}[scale=3.35]
\draw[thick,->] (-0.2,0) -- (1.2,0) node[below]{\(p\)};
\draw[thick,->] (0,-0.2) -- (0,1.2) node[left]{\(q\)};

\foreach \t / \ttext in {1, 0.5 / \frac12, 0.8 / \frac45} {
  \draw (\t,0.05) -- (\t,-0.05) node[below,fill=dgray,
        inner sep=0.3ex,text height=2.2ex]{$\ttext$};
  \draw (0.05,\t) -- (-0.05,\t) node[left, fill=dgray,
        inner sep=0.3ex,text height=2.2ex]{$\ttext$};}
  \draw (0,-0.05) node[below,fill=dgray,
        inner sep=0.3ex,text height=2.2ex]{$0$};

\draw[very thick,fill1] (0,1) -- (0.8,1) -- (0.8,0) -- (1,0);
\draw[very thick,fill3] (1,0) -- (1,0.8) -- (0,0.8) -- (0,1);

\foreach \x / \y in {0 / 1, 0.8 / 0.8, 1 / 0} {
   \fill[PaleVioletRed,fill opacity=0.85] (\x , \y) circle[radius=0.4mm];}
\end{tikzpicture} \end{center}
\end{frame}
