\documentclass[12pt,aspectratio=43,svgnames]{beamer}

\usepackage{modules/cgs}

\begin{document} \maketitle

\begin{frame} \ft{Содержание}
	\tableofcontents
\end{frame}

\section{Битва в море Бисмарка}

\begin{frame} \ft{Битва в море Бисмарка}
	Генерал Имамура может послать конвой\\
	северным маршрутом (2 дня) или\\
	южным маршрутом (3 дня).
	\medskip

	Генерал Кенни хочет бомбить конвой;\\
	если он отправит свои самолёты {\it не туда,}\\
	у него будет на это полдня меньше.\quad{\scriptsize\cite{petersGT}}
\end{frame}

\newcommand{\bismarckPO}{\gamePayoffs{2}{-2}{2.5}{-2.5}{1.5}{-1.5}{3}{-3}}

\begin{frame} \ft{Запись игры с помощью таблицы}
Кенни выбирает строку таблицы, Имамура выбирает\\
столбец. Их выигрыши записаны в соотв. клетках\\
таблицы напротив их выбора.
\begin{center} \begin{tikzpicture}[scale=0.44]
	\gameLines
	\gameNames{Кенни}{Имамура}
	\gameActions{Север}{Юг}
	\bismarckPO	
\end{tikzpicture} \end{center}
\end{frame}

\begin{frame} \ft{Доминирующая стратегия}
При любом действии Кенни Имамуре выгоднее\\
выбирать север (см. строчки).\\
У Имамуры есть домин.\:стратегия, у Кенни нет.
\begin{center} \begin{tikzpicture}[scale=0.44]
	\gameLines
	\gameNames{Кенни}{Имамура}
	\gameActions{Север}{Юг}
	\bismarckPO
	\obvod{-1.5}{3}{3.5}{3}{fill1}{0.5}{0}{\geqslant}
	\obvod{-1.5}{-1}{3.5}{-1}{fill1}{0.5}{0}{\geqslant}
\end{tikzpicture} \end{center}
\end{frame}

\begin{frame} \ft{Равновесие Нэша}
Зная это, Кенни тоже выберет север. Позиция\\
\(\lr*{\text{Север},\text{Север}}\)~— {\it равновесие Нэша}: действие \\
каждого~— лучший ответ на действие другого.
\begin{center} \begin{tikzpicture}[scale=0.44]
	\fill[p9480u,opacity=0.3] (0,0) rectangle (-5,4);
	\gameLines
	\gameNames{Кенни}{Имамура}
	\gameActions{Север}{Юг}
	\bismarckPO
	\obvod{-1.5}{3}{3.5}{3}{fill1}{0.5}{0}{\geqslant}
	\obvod{-3.5}{-3}{-3.5}{1}{fill3}{0.5}{-90}{\geqslant}
\end{tikzpicture} \end{center}
\end{frame}

\begin{frame} \ft{Равновесие Нэша}
	Равновесие Нэша~— это {\it устойчивое состояние\\
	общества,} такой закон, который никто\\
	не будет хотеть нарушить даже при\\
	отсутствии какого-либо контроля.
\end{frame}

\section{Золотые шары (дилемма заключённого)}

\begin{frame} \ft{Что такое дилемма заключённого?}
	Известная игра, где равновесие Нэша\\
	находится не в позиции, которая\\
	кажется предпочтительной\\
	для обоих игроков.\quad{\scriptsize\cite{petersGT,mszGT}}
	\medskip

	Адаптирована в качестве телешоу\\
	«Золотые шары» на британском\\
	канале {\it ITV.}\quad{\scriptsize\cite{zurichGB}}
\end{frame}

\newcommand{\gbPO}{\gamePayoffs{5}{5}{0}{10}{10}{0}{0}{0}}

\begin{frame} \ft{Таблица выигрышей для «З.~Ш.»}
	Оба делятся~— выигрыш делится поровну.\\
	Один делится~— всё забирает другой.\\
	Оба хотят забрать~— остаются ни с чем.
\begin{center} \begin{tikzpicture}[scale=0.44]
	\gameLines
	\gameNames{Игрок 1}{Игрок 2}
	\gameActions{Делить}{Забрать}
	\gbPO
\end{tikzpicture} \end{center}
\end{frame}

\begin{frame} \ft{Что тут происходит?}
	У обоих игроков есть доминирующая\\
	стратегия: забирать деньги.\\
	Она всегда даёт не меньший выигрыш.
\begin{center} \begin{tikzpicture}[scale=0.44]
	\gameLines
	\gameNames{Игрок 1}{Игрок 2}
	\gameActions{Делить}{Забрать}
	\gbPO
	\obvod{-1.5}{3}{3.5}{3}{fill1}{0.5}{0}{\leqslant}
	\obvod{-1.5}{-1}{3.5}{-1}{fill1}{0.5}{0}{\leqslant}
\end{tikzpicture} \end{center}
\end{frame}

\begin{frame} \ft{Равновесие Нэша}
	В этой игре три равновесия Нэша,\\
	но ни одно из них~— не\\
	\(\lr*{\text{Делить},\text{Делить}}\).
\begin{center} \begin{tikzpicture}[scale=0.44]
	\fill[p9480u,opacity=0.3] (0,0) rectangle (5,-4)
		(0,0) rectangle (5,4) (0,0) rectangle (-5,-4);
	\gameLines
	\gameNames{Игрок 1}{Игрок 2}
	\gameActions{Делить}{Забрать}
	\gbPO
	\obvod{-1.5}{-1}{3.5}{-1}{fill1}{0.44}{0}{=}
	\obvod{-1.5}{3}{3.5}{3}{fill1}{0.44}{0}{\leqslant}
	\obvod{1.5}{-3}{1.5}{1}{fill3}{0.44}{-90}{=}
	\obvod{-3.5}{-3}{-3.5}{1}{fill3}{0.44}{-90}{\leqslant}
\end{tikzpicture} \end{center}
\end{frame}

\begin{frame} \ft{Парето-оптимум}
	Участники пытаются разработать такую\\
	систему контроля, которая бы заставила их\\
	гарантированно находиться в позиции,\\
	оптимальной по Парето:
	\medskip
	
	Нельзя улучшить чей-либо выигрыш,\\
	не ухудшив суммарного выигрыша и\\
	справедливости его распределения.
\end{frame}


\begin{frame} \scriptsize
	\bibliography{lile}
	\bibliographystyle{apalike}
\end{frame}

\end{document}