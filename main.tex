\documentclass[12pt,aspectratio=43,svgnames]{beamer}

\usepackage{modules/cgs}

\begin{document} \maketitle

\begin{frame} \ft{Содержание}
	\tableofcontents
\end{frame}

\section{Битва в море Бисмарка}

\begin{frame} \ft{Битва в море Бисмарка}
	Генерал Имамура может послать конвой\\
	северным маршрутом (2 дня) или\\
	южным маршрутом (3 дня).

	Генерал Кенни хочет бомбить конвой;\\
	если он отправит свои самолёты {\it не туда,}\\
	у него будет на это полдня меньше.\quad{\scriptsize\cite{petersGT}}
\end{frame}

\newcommand{\bismarckPO}{\gamePayoffs{2}{-2}{2.5}{-2.5}{1.5}{-1.5}{3}{-3}}

\begin{frame} \ft{Запись игры с помощью таблицы}
Кенни выбирает строку таблицы, Имамура выбирает\\
столбец. Их выигрыши записаны в соотв. клетках\\
таблицы напротив их выбора.
\begin{center} \begin{tikzpicture}[scale=0.44]
	\gameLines
	\gameNames{Кенни}{Имамура}
	\gameActions{Север}{Юг}
	\bismarckPO	
\end{tikzpicture} \end{center}
\end{frame}

\begin{frame} \ft{Доминирующая стратегия}
При любом действии Кенни Имамуре выгоднее\\
выбирать север (см. строчки).\\
У Имамуры есть домин.\:стратегия, у Кенни нет.
\begin{center} \begin{tikzpicture}[scale=0.44]
	\gameLines
	\gameNames{Кенни}{Имамура}
	\gameActions{Север}{Юг}
	\bismarckPO
	\obvod{-1.5}{3}{3.5}{3}{fill1}{0.5}{0}{\geq}
	\obvod{-1.5}{-1}{3.5}{-1}{fill1}{0.5}{0}{\geq}
\end{tikzpicture} \end{center}
\end{frame}

\begin{frame} \ft{Равновесие Нэша}
Зная это, Кенни тоже выберет север. Позиция\\
\(\lr*{\text{Север},\text{Север}}\)~— {\it равновесие Нэша}: действие \\
каждого~— лучший ответ на действие другого.
\begin{center} \begin{tikzpicture}[scale=0.44]
	\fill[p9480u,opacity=0.37] (0,0) rectangle (-5,4);
	\gameLines
	\gameNames{Кенни}{Имамура}
	\gameActions{Север}{Юг}
	\bismarckPO
	\obvod{-1.5}{3}{3.5}{3}{fill1}{0.5}{0}{\geq}
	\obvod{-3.5}{-3}{-3.5}{1}{fill3}{0.5}{-90}{\geq}
\end{tikzpicture} \end{center}
\end{frame}

\begin{frame} \ft{Равновесие Нэша}
	Равновесие Нэша~— это {\it устойчивое состояние\\
	общества,} такой закон, который никто\\
	не будет хотеть нарушить даже при\\
	отсутствии какого-либо контроля.
\end{frame}

\begin{frame} \ft{Цитирование}
	\cite{petersGT,mszGT,brusselsEVO,derivativeEVO}
\end{frame}

\begin{frame} \scriptsize
	\bibliography{lile}
	\bibliographystyle{apalike}
\end{frame}

\end{document}