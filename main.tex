\documentclass[12pt,aspectratio=43,svgnames]{beamer}

\usepackage{modules/cgs}

\begin{document} \maketitle

\begin{frame} \ft{Содержание}
	\tableofcontents
\end{frame}

\section{Битва в море Бисмарка}

\begin{frame} \ft{Битва в море Бисмарка}
Sample text.~{\scriptsize\cite{petersGT}}
\end{frame}

\newcommand{\bismarckPO}{\gamePayoffs{2}{-2}{2.5}{-2.5}{1.5}{-1.5}{3}{-3}}

\begin{frame} \ft{Запись игры с помощью таблицы}
Кенни выбирает строку таблицы, Имамура выбирает\\
столбец. Их выигрыши записаны в соотв. клетках\\
таблицы напротив их выбора.
\begin{center} \begin{tikzpicture}[scale=0.44]
	\gameLines
	\gameNames{Кенни}{Имамура}
	\gameActions{Север}{Юг}
	\bismarckPO	
\end{tikzpicture} \end{center}
\end{frame}

\begin{frame} \ft{Доминирующая стратегия}
При любом действии Кенни\\
Имамуре выгоднее выбирать север:\\
посмотрим на числа в строчках.
\begin{center} \begin{tikzpicture}[scale=0.44]
	\gameLines
	\gameNames{Кенни}{Имамура}
	\gameActions{Север}{Юг}
	\bismarckPO
	\obvod{-1.5}{3}{3.5}{3}{fill1}{0.5}{0}{\leq}
	\obvod{-1.5}{-1}{3.5}{-1}{fill1}{0.5}{0}{\leq}
\end{tikzpicture} \end{center}
\end{frame}

\begin{frame} \ft{Цитирование}
	\cite{petersGT,mszGT,brusselsEVO,derivativeEVO}
\end{frame}

\begin{frame} \scriptsize
	\bibliography{lile}
	\bibliographystyle{apalike}
\end{frame}

\end{document}