\documentclass[11pt,aspectratio=169,svgnames]{beamer}

\usefonttheme{professionalfonts}

\usepackage{amsmath,amssymb,amsthm,mathtools}
\usepackage{xcolor,graphicx,tikz}
\usepackage[russian]{babel}

\definecolor{dgray}{RGB}{15,15,15}
\definecolor{dplot}{HTML}{96e6ff}

\usebackgroundtemplate{%
	\includegraphics[width=\paperwidth,height=\paperheight]{img/lile-back-dgray}%
}

\setbeamersize{text margin left=12mm,text margin right=12mm}
\addtolength{\headsep}{0.55cm}
\setbeamertemplate{frametitle}[default][left,leftskip=0.85cm]
\setbeamertemplate{navigation symbols}{}
\setbeamertemplate{blocks}[rounded]
\setbeamertemplate{footline}{\vspace{1.4cm}}

\setbeamercolor{titlelike}{fg=white}
\setbeamercolor{normal text}{fg=white}
\setbeamercolor{block title}{bg=white!27!dgray,fg=white}
\setbeamercolor{block body}{bg=white!10!dgray}

\DeclarePairedDelimiter{\lr}{(}{)}
\DeclarePairedDelimiter{\atdeg}{[}{]}
\newcommand{\br}{\mathbb{R}}

\usepackage{mathspec}

\setsansfont[
	Path = f/,
	Extension = .otf,
	BoldFont=fb,
	BoldItalicFont=fbi
		]{f}
		
\setmathfont(Digits)[Path = f/]{roboto.ttf}
\setmathfont(Latin)[Path = f/]{robotoi.ttf}
\setmathfont(Greek)[Path = f/, Uppercase]{roboto.ttf}
\setmathfont(Greek)[Path = f/, Lowercase]{robotoi.ttf}


\newenvironment{nblock}[1]{
	\begin{center} \begin{columns}[t] \begin{column}{110mm} \begin{block}{#1}
   }{
	\end{block} \end{column} \end{columns} \end{center}
}

\newcommand{\incg}[1]{\includegraphics[width=0.88\textwidth]{#1}}

   \title{О (дискретном) преобразовании Фурье}
   \date{\today}
   \author{Золотов Борис Алексеевич, аспирант МКН СПбГУ, \\ преподаватель ЛНМО}
   \institute{«Лига Лекторов», 3 сезон, онлайн-этап}

\begin{document} \maketitle

\begin{frame} \frametitle{Простое контрастное изображение}
	\begin{center} \incg{img/ndlg.pdf} \end{center}
	\phantom{И как это связано со способностью} \\
	\phantom{слышать одного своего друга в толпе?}
\end{frame}

\begin{frame} \frametitle{При сохранении в jpg «идёт волнами». Почему так?}
	\begin{center} \incg{img/ndlg.jpg} \end{center}
	И как это связано со способностью \\
	слышать одного своего друга в толпе?
\end{frame}


\begin{frame} \frametitle{\vspace*{-2.4cm}}
    \incg{python-fourier/signal}
\end{frame}

\begin{frame} \frametitle{\vspace*{-2.4cm}}
    \incg{python-fourier/waveform-2}

    \incg{python-fourier/fourier-2}
\end{frame}

\begin{frame} \frametitle{\vspace*{-2.4cm}}
    \incg{python-fourier/waveform-3}

    \incg{python-fourier/fourier-3}
\end{frame}

\begin{frame} \frametitle{\vspace*{-2.4cm}}
    \incg{python-fourier/waveform-5}

    \incg{python-fourier/fourier-5}
\end{frame}

\begin{frame} \frametitle{\vspace*{-2.4cm}}
    \incg{python-fourier/waveform-9}

    \incg{python-fourier/fourier-9}
\end{frame}

\begin{frame} \frametitle{\vspace*{-2.4cm}}
    \incg{python-fourier/waveform-15}

    \incg{python-fourier/fourier-15}
\end{frame}

\begin{frame} \frametitle{\vspace*{-2.4cm}}
    \incg{python-fourier/waveform-23}

    \incg{python-fourier/fourier-23}
\end{frame}

\begin{frame} \frametitle{\vspace*{-2.4cm}}
    \incg{python-fourier/waveform-25}

    \incg{python-fourier/fourier-25}
\end{frame}




\end{document}

\begin{frame} \frametitle{}
\end{frame}

\begin{nblock}{\vspace*{-3ex}}
	Sample text
\end{nblock}
